\documentclass[10pt]{article}
\usepackage{bold-extra} % for bold \texttt

\usepackage{../xreview}

\newcommand{\writecommand}[1]{\texttt{\textbf{{\textbackslash#1}}}}
\newcommand{\writeenv}[1]{\texttt{\textbf{{#1}}}}
\newcommand{\writearg}[1]{\{#1\}}

\title{XReview: Authored Comments}
\date{}

\setlength{\marginparwidth}{3 cm}

\begin{document}

\maketitle

It is most times helpful to know who wrote a comment. This is made easy by the \writecommand{authoredcomments} command. This is used as such 

\begin{quote}
    \writecommand{authoredcomments}[Optional: mods to \writecommand{comment}][Optional: mods to \writecommand{resolvedcomment}]{Authorname}
\end{quote}
\noindent
and creates two commands \writecommand{Authornamecomment} and \writecommand{resolvedAuthornamecomment} which will mark the author and can easily be distinguished.
Here is an example:
~\newline

\textbf{\LaTeX Code}

\begin{quote}
    \writecommand{authoredcomments}[color=cyan]\writearg{Lorenzo}\newline
    \writecommand{authoredcomments}[color=orange]\writearg{Alice}\newline

    This is \writecommand{Alicecomment}\writearg{great}\writearg{Written by Alice.} while this \writecommand{Lorenzocomment}\writearg{less so}\writearg{Written by Lorenzo}.

    This is an \writecommand{resolvedLorenzocomment}\writearg{old modification}\writearg{Found appropriate citation.} we discussed.
\end{quote}

\writecommand{showchanges}
\showchanges

\begin{quote}
    \authoredcomments[color=cyan]{Lorenzo}
    \authoredcomments[color=orange]{Alice}
This is \Alicecomment{great}{Written by Alice.} while this \Lorenzocomment{less so}{Written by Lorenzo}.

This is an \resolvedLorenzocomment{old modification}{Found appropriate citation.} we discussed.
\end{quote}

\end{document}