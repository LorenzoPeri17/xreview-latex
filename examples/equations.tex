\documentclass[10pt]{article}
\usepackage{bold-extra} % for bold \texttt

\usepackage{../xreview}

\newcommand{\writecommand}[1]{\texttt{\textbf{{\textbackslash#1}}}}
\newcommand{\writeenv}[1]{\texttt{\textbf{{#1}}}}
\newcommand{\writearg}[1]{\{#1\}}

\title{XReview: Equations}
\date{}

\begin{document}

\maketitle

To deal with removed and added equations, \textit{xreview} provides the \writeenv{remequation}, \writeenv{remequation*}, \writeenv{addequation}, \writeenv{addequation*} environments.

\writeenv{remequation} and \writeenv{remequation*} typeset equations as if they were \writecommand{removed} and \textbf{are indistinguishible from \writeenv{equation} and \writeenv{equation*} in the clean version}.

\writeenv{addequation} and \writeenv{addequation*} typeset equations as if they were \writecommand{added} and \textbf{hide them from the clean version}.

\textit{xreview} also takes care of properly deal with equation numbering, to ensure that \textbf{equation numbering will not change between the annotated and clean versions}.

\begin{itemize}
    \item \writeenv{addequation*} and \writeenv{remequation*} do not number the equation;
    \item \writeenv{addequation} follows on from regular equation numbering;
    \item \writeenv{remequation} introduces \textbf{a separate numbering scheme} for removed equations that one still wishes to reference.
\end{itemize}

The prefix for removed equation numbering is normally a capital `R'. However, this can be customized to one's preference. These labels are fully compatible with referencing commands such as \writecommand{label}, \writecommand{ref}, and \writecommand{eqref}.

\begin{center}
    \begin{minipage}[t]{0.3\linewidth}
    \textbf{\LaTeX Code}
    
    This equation is kept \newline
    \writecommand{begin}\writearg{equation}

    ~\quad a+b=c\newline
    \writecommand{end}\writearg{equation}\newline
    while this is removed and labelled\newline
    \writecommand{begin}\writearg{remequation}

    ~\quad a+b = d\newline
    \writecommand{label}\writearg{eq:remeq}\newline
    \writecommand{end}\writearg{remequation}\newline
    without altering the normal equation numbering.\newline
    \writecommand{begin}\writearg{equation}

    ~\quad b+c = d\newline
    \writecommand{end}\writearg{equation}\newline
    This is added instead
    \writecommand{begin}\writearg{addequaiton}

    ~\quad b = 3\newline
    \writecommand{end}\writearg{addequation}\newline

    \end{minipage}
    \hfill
    \begin{minipage}[t]{0.3\linewidth}
    \writecommand{showchanges}
    \showchanges
    
    This equation is kept
    \begin{equation}
        a+b = c
    \end{equation}
    while this is removed and labelled
    \begin{remequation}
        a+b = d
        \label{eq:remeq}
    \end{remequation}\addtocounter{removedeqcounter}{-1}
    without altering the normal equation numbering.
    \begin{equation}
        b+c = d
    \end{equation}
    This is added instead
    \begin{addequation}
        b=3
    \end{addequation}\addtocounter{equation}{-3}

    \end{minipage}
    \hfill
    \begin{minipage}[t]{0.3\linewidth}   
    \writecommand{showclean}
    \showclean
    
    This equation is kept
    \begin{equation}
        a+b = c
    \end{equation}
    while this is removed and labelled
    \begin{remequation}
        a+b = d
    \end{remequation}\addtocounter{removedeqcounter}{-1}without altering the normal equation numbering.
    \begin{equation}
        b+c = d
    \end{equation}
    This is added instead
    \begin{addequation}
        b=3
    \end{addequation}\addtocounter{equation}{-3}

\end{minipage}
\end{center}
\showchanges

Now I can reference the removed Eq.~\eqref{eq:remeq} via \writecommand{eqref}\writearg{eq:remeq}.

\end{document}