\documentclass[10pt]{article}
\usepackage{bold-extra} % for bold \texttt
\usepackage{hyperref}

\usepackage{../xreview}

\newcommand{\writecommand}[1]{\texttt{\textbf{{\textbackslash#1}}}}
\newcommand{\writeenv}[1]{\texttt{\textbf{{#1}}}}
\newcommand{\writearg}[1]{\{#1\}}

\title{XReview: Comments}
\date{}

\begin{document}

\maketitle

Comments are a great way to communicate with various authors, to carry out a discussion directly on a shared document or to share ideas and questions together with the compiled manuscript.

\textit{xreview} allows for the possibility to insert comments that out of the box work in one- or two-column documents. 

Simple comments can be created with the \writecommand{comment} command. They highlight a snippet of text and \textbf{are hidden on the clean version}. 
~\newline

\textbf{\LaTeX Code}

\begin{quote}
\writecommand{textit}\writearg{xreview} is \writecommand{comment}\writearg{the greatest \writecommand{TeX} package ever written!}\writearg{Have you heard of \writecommand{textit}\writearg{amsmath}?}
\end{quote}

\writecommand{showchanges}
\showchanges

\begin{quote}
\textit{xreview} is \comment{the greatest \TeX~package ever written!}{Have you heard of \textit{amsmath}?}
\end{quote}

\writecommand{showclean}
\showclean

\begin{quote}
\textit{xreview} is \comment{the greatest \TeX~package ever written!}{Have you heard of \textit{amsmath}?}
\end{quote} \addtocounter{commentcounter}{-1}

\showchanges

\textbf{Every comment is numbered} for ease of reference, with a customizable prefix (a capital `C' by default, see Section~\ref{sec:custom}).

Sometimes, it is desirable to share only the changes and hide the comments. For this scenario, \textit{xreview} provides a separate toggle:
\begin{itemize}
    \item \writecommand{showcomments} to show comments on the annotated document;
    \item \writecommand{hidecomments} to hide comments \textbf{even if \writecommand{showchanges} is set}.
\end{itemize}

\writecommand{comment} takes as optional argument everything accepted by \writecommand{todo} in the \href{https://ctan.org/pkg/todonotes?lang=en}{todonotes package}. This can be used, for example, to override the default color.
~\newline

\textbf{\LaTeX Code}

\begin{quote}
    This is a \writecommand{comment}[backgroundcolor=red]\writearg{red comment!}\writearg{It looks angry.}
\end{quote}

\writecommand{showchanges}
\showchanges

\begin{quote}
This is a \comment[backgroundcolor=red]{red comment!}{It looks angry.}
\end{quote}

\end{document}